\begin{savequote}[45mm]
\ascii{Make it work, make it right, make it fast.}
\qauthor{\ascii{- Kent Beck}}
\end{savequote}

\chapter{General Programming}
\label{ch:general-programming}

\section{Code Style}

\begin{content}

\begin{regulation}
团队应该统一代码风格
\end{regulation}

团队代码拥有一种最权威的风格,犹如一个人写的一样。否则代码因缺乏统一整齐的花括号,或者存在长度不一致的\ascii{TAB},使得代码变得层次不齐。

存在多种经典的代码风格,它们都拥有各自的特点和优势,团队应该统一地选择其中一种即可。
\begin{enum}
  \eitem{\ascii{K\&R}}
  \eitem{\ascii{BSD/Allman}}
  \eitem{\ascii{GNU}}
  \eitem{\ascii{Whitesmiths}}
\end{enum}

统一代码风格,并非难事,团队发布统一的\ascii{IDE}代码模板,及其定制一个方便的格式化快捷键即可解决所有问题。

\begin{regulation}
拒绝实现巨文件
\end{regulation}

巨大的头文件最大的一个问题就是产生了难以忍受的编译时依赖,这种往往是由\ascii{C}遗留的一些陋习所致,拒绝它的诱惑吧。

巨大的实现文件虽然没有编译时依赖的问题,但它给人带来巨大的恐惧感;此外在一个\ascii{100000}行的源文件中进行重构简直就是一件痴人说梦的事情;在多人同时修改此文件的概率大大加大,使用配置管理工具时冲突不断。

多少行是最适合的呢,没有标准,犹如一个函数应该有多少行一样的问题;只有一个高内聚、低耦合的原则可参考;但事实证明,往往满足了这个原则的时候,文件就变得短小精干了。

\end{content}


\section{Clean Comment}

\begin{content}

\begin{regulation}
消除所有没有必要的注释
\end{regulation}

没有携带任何信息量的注释都是没有必要的。

如下列的注释,维护它简直就是一种巨大的包袱,有时候它的存在就像一个笑话。它就像挑战读者的智商,如果你不知道它是一个构造函数,让我告诉你吧。

\begin{leftbar}
\begin{c++}
/**
 * Constructor
 */
InvokedAtMost(const unsigned int times);

/**
 * @param Invocation
 * @return bool
 */
bool matches(const Invocation& inv) const;
\end{c++}
\end{leftbar}

还有一种就是误导性的注释,注释和代码已经失去匹配,意图甚至是相反的。这样的注释如果继续存在,贻害的肯定不止一个人。

还有一种注释会经常遇到:日志型注释。修正一个\ascii{bug}时,都需要在函数头的注释表中增加一条记录.放弃这种好习惯,将这份精细的工作交给源代码控制系统吧。

还有一种注释,当我遇到时,就会产生一种莫名的冲动将其消灭。在花括号后面的\ascii{//end if, //end while, // end try},它们的出现就是提取函数的最佳时机。

代码签署问题,往往是因为版权等问题被强制要求。至于我从来不会将自己的名字留在代码中,也许只有代码巨匠才敢做这件事情。事实上,留了名字的代码往往只会给后来人唯一的一个提示就是:这个代码只有我一个人能维护,别抢我的饭碗。

已经被注释掉的代码,应该立即删除。因为从源代码控制系统中追回这份被删除的代码是很容易的。

\begin{regulation}
在需要注释的时候,一定要加注释
\end{regulation}

法律信息注释,这是因为公司版权的保护不得不让程序员增加这些注释。这类问题幸好可以通过代码模板轻松解决,不会成为造成程序员的任何负担\footnote{发布开源代码时,也常常需要增加必要的\ascii{GNU, BSD等}公共许可证};

对代码无法明确的意图也需要增加注释,例如对正则表达式的注释,位移操作的注释等。

\begin{leftbar}
\begin{c++}
// kk::mm::ss, MM dd, yyyy
std::string timePattern = "\\d{2}:\\d{2}:\\d{2}, \\d{2} \\d{2}, \\d{4}";
\end{c++}
\end{leftbar}

当在代码中存在反直觉,和可能存在陷阱的代码也需要增加注释。这类情况,首先应该考虑能否消除这样的陷阱,但有时候是很难做到,注释可能是最后的一根救命稻草了。

\end{content}

\section{General Rules}

\begin{content}

\begin{regulation}
绝不使用全局变量
\end{regulation}

全局变量犹如\ascii{goto}一样臭名昭著,但程序员常常被全局变量所诱惑,因为它的确容易被实现。但是它带来了破坏性远远大于其益处。作为一个团队,应该拥有统一的编码规程,禁止使用全局变量应该成为团队中大家必须遵守一条法文。

太苛刻了,但从实践情况上了,这样的要求并没有想象中那么苛刻,总会找到一种比全局变量更好的解决方案。

在最坏的情况下,当需要给系统唯一存在的概念进行描述,也不应该定义全局变量,而应该使用单态进行包装。

\begin{regulation}
只有在简单逻辑的情况下鼓励使用\ascii{?:}条件表达式。
\end{regulation}

\begin{leftbar}
\begin{c++}
if (hour >= 12) 
{
    time_str += "pm";
} 
else 
{
    time_str += "am";
}
\end{c++}
\end{leftbar}

这样的语句鼓励重构为:
\begin{leftbar}
\begin{c++}
time_str += (hour >= 12) ? "pm" : "am";
\end{c++}
\end{leftbar}

虽然道理简单,但很多程序员往往因为习惯了\ascii{if-else}而忽视了这种简洁的表达式。犹如常常看到代码中会有如下的语句一样,让人哭笑不得。

\begin{leftbar}
\begin{c++}
bool exist = erabs.contains(ErabId(2)) ? true : false;
\end{c++}
\end{leftbar}

但是,当表达式变得非常复杂时,\ascii{if-else}可能是更好的选择了。

\begin{leftbar}
\begin{c++}
return exponent >= 0 ? mantissa * (1 << exponent) : mantissa / (1 << -exponent);
\end{c++}
\end{leftbar}

\begin{regulation}
擅用卫语句从函数中提前返回,或从嵌套的语句中提前返回。
\end{regulation}

\begin{leftbar}
\begin{c++}
bool contains(const std::string* str, const string* substr) 
{
    if (str == NULL || substr == NULL)
        return false;
    
    if (substr->empty()) 
        return true;

    ...
}
\end{c++}
\end{leftbar}

可以想象,如果不进行提前返回,上述处理的表达效果将大打折扣。

虽然在代码阅读过程中\ascii{continue}常常打断了人的思维。但在函数短小,意图明确的情况下,\ascii{continue}相反能给人一种“跳过此项”的意图,代码嵌套逻辑得到了改善。

\ascii{Dijkstra}提出了结构化编程规则,我们赞同结构编程的目标和规范。但对于小函数,这条规则指导意义不是很大。

\begin{regulation}
擅用解释型变量,或提供意图明确的、内联的查询函数。
\end{regulation}

解释型变量的重构手法因为内联的查询函数的威力所大大折扣,往往没有后者运用广泛。

\begin{leftbar}
\begin{c++}
if (line.split(":")[0].trim() == "root")
\end{c++}
\end{leftbar}

\begin{leftbar}
\begin{c++}
String userName = line.split(":")[0].trim();
if (userName == "root")
{
    ...
}
\end{c++}
\end{leftbar}

\ascii{split}函数将字符串按照指定的正则表达式进行拆分,并提供一个良好的意图变量,将大大改善逻辑的清晰度。

\begin{regulation}
如果需要精确答案,请避免使用\ascii{float}和\ascii{double}
\end{regulation}

由于二进制浮点运算只是一种快速的近似的数值运算计算,所以在精度上可能与想象的有一定误差。常见的做法是通过扩大倍数,将浮点数运算转换为整数运算;或直接利用支持高精度浮点运算的类库。

\begin{regulation}
复用类库,复用既有代码。
\end{regulation}

复用类库,复用既有代码,而不是重新造一个轮子,这给程序员提出了一个很大的挑战。就像学习\ascii{ruby}一样,入门简单,但真正的精通,还是得对库的熟悉运用程度。

\end{content}
