\begin{savequote}[45mm]
\ascii{Premature optimization is the root of all evil.}
\qauthor{\ascii{– Donald Knuth}}

\ascii{On the other hand, we cannot ignore efficiency.}
\qauthor{\ascii{– Jon Bentley}}
\end{savequote}

\chapter{Functions and Operators}
\label{ch:functions-operators}

%%%%%%%%%%%%%%%%%%%%%%%%%%%%%%%%%%%%%%%%%%%%%%%%%%%%%%%%%%%%%%%%%%%%%%%%%%%%%%%%
\section{Function}

\begin{content}

\begin{regulation}
避免过长,嵌套过深的函数实现
\end{regulation}

我讨厌长函数,犹如讨厌重复一样。长函数往往伴随着复杂的逻辑判断,过深嵌套逻辑。理解它们的业务规则是一件痛苦的事情。

\begin{regulation}
只做一件事,并做好这件事
\end{regulation}

这是\ascii{SRP}在函数实现中的一个具体体现。一个函数只应该承担一个唯一的职责,如果函数名伴随\ascii{and},或将查询和命令混合往往是违背此原则的信号。

一般地,如果函数只是做了该函数名称下统一抽象层次上的几个小步骤,则函数还是只做了一件事情。函数就是将一个大一点的概念拆分成级别稍微低一点、并在同一个抽象层次的一系列步骤的过程。

\begin{regulation}
函数中的每一个语句都在一个相同的抽象层次上
\end{regulation}

如果函数中混杂不同抽象层次的代码,往往让人感到迷惑,理解代码的逻辑,需要读者陷入到细节之中而不能自拨。

\ascii{Kent Beck}建议使用\ascii{Compose Method}分解长函数;\ascii{Martin Flower}也建议使用\ascii{Extract Method}进行函数提取。\ascii{Extract Method}最重要的就是识别出代码中的抽象层次,并梳理出主干,按照自顶向下的规则实现函数的。

函数提取常常要遵守如下3个基本原则:
\begin{enum}
  \eitem{\ascii{在同一个抽象层次}}
  \eitem{\ascii{给一个直观的,意图明确的好名字}}
  \eitem{\ascii{实现对称性}}
\end{enum}

正如\ascii{Bob}大叔提到的那样: 程序就像是一系列的以\ascii{TO}起头的段落,每一段落都描述了当前抽象层次的逻辑,并引用下一抽象层次的,以TO开头的段落。

下例虽然以\ascii{Java}为例,但重点没有偏离,规则同样适用于\ascii{C/C++}。

\begin{leftbar}
\begin{java}
public class List<E> {
   public void add(E element) {
      if (!readOnly) {
         int newSize = size + 1;
         if (newSize > elements.length) {
            Object[] newElements =
               new Object[elements.length + 10];
            for (int i = 0; i < size; i++)
               newElements[i] = elements[i];
            elements = newElements;
         }
         elements[size++] = element;
      }
   }
}
\end{java}
\end{leftbar}

提取函数之后,使算法的骨骼显而易见。

\begin{leftbar}
\begin{java}
public class List<E> {
   public void add(E element) {
      if (readOnly)
         return;
      
      if (atCapacity())
         grow();
      
      addElement(element);
   }
}
\end{java}
\end{leftbar}

提取函数往往受到性能偏执狂的抨击,他们认为过多的函数提取将成为性能的主要瓶颈。\ascii{2-8}原则将彻底击垮所有不靠谱的言论,与其纠结于函数调用的开销,不如花费更多的时间去优化和改善那些至关重要的算法和关键代码。

\begin{regulation}
检查参数的有效性
\end{regulation}

绝大部分的函数对于传递的参数都有限制,这时需要在函数执行之前完成参数的合法性校验。\ascii{C/C++}语言不能天然地支持“按契约设计”的语言,但使用前置的参数检查将大大改善程序的健壮性。

\begin{regulation}
将局部变量的作用域最小化
\end{regulation}

这条规则可能对\ascii{C}程序员更有指导意义,但改变这样的陋习可能需要一段时间。例如局部于\ascii{for}的循环变量,当它的使命完成之后,强制让编译器将其销毁,避免这个变量再次被使用的风险。

\begin{leftbar}
\begin{c++}
map<string, string> addressBook; 
for ( auto entry : addressBook ) 
{ 
    cout << entry.first << "," << entry.second << endl;
}
\end{c++}
\end{leftbar}

同样的规则也使用与其他常见的情况,例如\ascii{if}语句,简化空指针的判断,简洁有效。

\begin{leftbar}
\begin{c++}
if (PaymentInfo* info = database.ReadPaymentInfo()) 
{
    cout << "User paid: " << info->amount() << endl;
}
\end{c++}
\end{leftbar}

但也有人对于\ascii{if}语句这样的用法持反对态度,因为他们认为这样做更容易出错。

\begin{regulation}
分离查询与指令
\end{regulation}

函数应该只拥有清晰明确的、唯一的职责。如果函数既修改对象状态,又返回对象的状态信息,则常常会导致混乱。产生副作用的指令操作,都是系统状态的一种变更,将产生时序上的耦合和顺序的依赖。尤其当查询和指令混合在一起的时候,查询函数的无副作用的优点将不复存在。

\begin{regulation}
使用\ascii{Null Object}替代空指针
\end{regulation}

校验空指针是一件及其繁琐的事情,优秀的程序员通过各种技巧绕开这些冗余的操作。例如参数通过引用传递,另外一种常见的手段就是\ascii{Null Object}\footnote{参考\ascii{Martin Flower}的著作《重构,改善既有代码的设计》}。

\begin{regulation}
对于参数类型,返回值类型,优先使用接口类型
\end{regulation}

这是基于接口编程的良好习惯,是优秀设计的体现。

\begin{regulation}
对于\ascii{bool}参数,优先使用两个元素的枚举类型
\end{regulation}

从直观上,

\begin{leftbar}
\begin{c++}
Thermometer::newInstance(CELSIUS);
\end{c++}
\end{leftbar}

要比

\begin{leftbar}
\begin{c++}
Thermometer::newInstance(true);
\end{c++}
\end{leftbar}

容易理解得多,但牺牲了一点点代码复用性。对于这个问题,可以通过内部的\ascii{private}函数来解决这个问题。也就是说,带\ascii{bool}参数的函数依然存在,只不过被\ascii{private},以便实现对枚举参数的函数的代码复用。

\begin{regulation}
永远不要导出具有相同参数数目的的重载方法
\end{regulation}

重载是一种编译时的多态。只要函数具有相同名字,但参数数目,类型不同,即为重载。但滥用往往会误导使用\ascii{API}的程序员,尤其在具有相同参数数目的重载方法时,程序员需要清晰地知道所有类型的隐式转换规则,即其重载匹配规则,这无疑是一种没必要的负担。

\begin{leftbar}
\begin{c++}
struct ObjectOutputStream
{
    void write(bool);
    void write(char);
    void write(short);
    void write(int);
    void write(long);
    void write(float);
    void write(double);
};
\end{c++}
\end{leftbar}

面对疑难的时候,抛弃重载往往能得到更不容易犯错的解决方案。

\begin{leftbar}
\begin{c++}
struct ObjectOutputStream
{
    void writeBool(bool);
    void writeChar(char);
    void writeShort(short);
    void writeInt(int);
    void writeLong(long);
    void writeFloat(float);
    void writeDouble(double);
};
\end{c++}
\end{leftbar}

\begin{regulation}
使用\ascii{explicit}显式地禁止类型的隐式转换
\end{regulation}

隐式类型转换往往无声无息地被执行,通过\ascii{explicit}便能通过编译器清晰地捕获的所有隐式转换的事件,以便供程序员进一步决策。

\begin{regulation}
避免实现火车失事的代码
\end{regulation}

\begin{leftbar}
\begin{java}
String outputDir = ctxt.getOptions().getScratchDir().getAbsolutePath();
\end{java}
\end{leftbar}

使用火车失事形容这段代码,非常贴切。如果其调用链如果某一环节出现了问题,则整个调用将出现失败。这类串联的调用违反了\ascii{Demeter}法则,\ascii{ctxt}对象包含了多个选项,每个选项中存在一个临时目录,每个目录都有一个绝对路径,所有的知识都毫无保留地暴露给了用户。

仔细分析一下代码,再看看其中一个模块是如何使用这个\ascii{outputDir}的。

\begin{leftbar}
\begin{java}
String outFile = outputDir + File.SEPERATOR + className + ".class";
BufferedOutputStream bos = new BufferedOutputStream(new FileOutputStream(outFile));
\end{java}
\end{leftbar}

所有的一切都是为了创建指定名称的临时文件,理想的实现应该将所有本应该隐藏的知识归并到\ascii{ctxt}中实现。

\begin{leftbar}
\begin{c++}
BufferedOutputStream bos = ctxt.createScratchFileStream(className);
\end{c++}
\end{leftbar}

\end{content}

\section{Operators}

\begin{content}

\begin{regulation}
重载运算符必须保持原有操作符的语义,
\end{regulation}

如果重载运算符改变了程序员对固有知识的理解,将加大放错的几率。

\begin{regulation}
谨慎地使用转换运算符
\end{regulation}

转换运算符是一种危险的操作,它常常无声无息地让人犯错。谨慎使用,并非绝不使用,在合适的情况下,使用装换运算符,能够得到更为简洁的代码。

\begin{regulation}
优先使用前置\ascii{++operator}
\end{regulation}

这是提升效率的一种举措,提倡使用前置的操作运算符。尤其在操作重载了\ascii{++operator}的类对象,例如迭代器,更应该使用\ascii{++operator}。

\begin{regulation}
使用\ascii{operator*, operator->}实现类的包装或修饰
\end{regulation}

\ascii{operator*, operator->}操作符是提供包装和修饰功能的特殊工具,是一种典型的间接技术\footnote{软件工程有一条黄金定律:任何问题都可以通过增加一个间接层来解决。}。它的最大优势是为用户提供良好的、人性化操作接口。

以扩展了的\ascii{placement new}为例,讲解\ascii{operator*, operator->}的使用。

\begin{leftbar}
\begin{c++}
#include <string.h>
#include <new>

template <typename T>
struct Placement
{
    void* alloc()
    {
        memset(u.buff, 0, sizeof(u));
        return (void*)u.buff;
    }

    void dealloc()
    {
        getObject()->~T();
    }    

    T* operator->() const
    {
        return (T*)u.buff;
    }

    T& operator*() const
    {
        return *(T*)u.buff;
    }

    T* getObject() const
    {
        return (T*)u.buff;
    }

private:
    union
    {
        char   c;
        short  s;
        int    i;
        long   l;
        float  f;
        double d;
        void*  p;

        char buff[sizeof(T)];
    }u;
};
\end{c++}
\end{leftbar}

\ascii{Placement}本质上就是一块内存,它是在原生内存上提供了一层直观的,人性化的操作接口。


众所周知,当定义一个对象的数组时,该对象的类必须提供一个默认构造函数。如下例,\ascii{ReleasingRbList}如果定义了一个\ascii{ReleasingRb},则此时因为\ascii{ReleasingRb}没有提供对应的默认构造函数而出现编译错误。

\ascii{Placement}能否发挥其优势,是解决此类问题的灵丹妙药;在需要创建对象的时候,再\ascii{placement new}出来,这也是一种典型的延迟初始化技术。

\begin{leftbar}
\begin{c++}
#include "base/Status.h"
#include "erab/ErabId.h"
#include "rb/RbId.h"
#include "base/Placement.h"

struct ReleasingRb
{
    ReleasingRb(const ErabId& erabId, const RbId& rbId);

    const ErabId& getErabId() const;
    const RbId& getRbId() const;

private:
    ErabId erabId;
    RbId   rbId;
};

struct ReleasingRbList
{
    ReleasingRbList();

    bool contains(const ErabId& erabId) const;
    Status addReleasingRb(const ErabId& erabId, const RbId& rbId);

private:
    enum { MAX_UE_ERAB_NUM = 8 };

    BYTE n;
    Placement<ReleasingRb> relErabs[MAX_UE_ERAB_NUM];
};
\end{c++}
\end{leftbar}

\begin{leftbar}
\begin{c++}
#include "erab/ReleasingRbList.h"
#include "base/Assertions.h"

ReleasingRbList::ReleasingRbList() : n(0)
{	
}

Status ReleasingRbList::addReleasingRb(const ErabId& erabId, const RbId& rbId)
{
    ASSERT_FALSE(contains(erabId));
    ASSERT_TURE(n < MAX_UE_ERAB_NUM);

    new (relErabs[n++].alloc()) ReleasingRb(erabId, rbId);

    return DCM_SUCCESS;
}

bool ReleasingRbList::contains(const ErabId& erabId) const
{
    for (BYTE i = 0; i < n; ++i)
    {
        if(relErabs[i]->getErabId() == erabId)
        {
            return true;
        }
    }

    return false;
}
\end{c++}
\end{leftbar}

\ascii{relErabs[i]->getErabId()}调用时,实际上调用了\ascii{Placement::operator->},但从实现上看,给用户调用提供了很大的灵活性和方便性,这还是得倚仗了\ascii{operator->}带来的间接性的功劳。

\end{content}
