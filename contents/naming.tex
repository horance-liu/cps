\begin{savequote}[45mm]
\ascii{Write programs for people first, computers second.}
\qauthor{\ascii{- Steve McConnell}}
\end{savequote}

\chapter{Naming} 
\label{ch:naming}

\begin{content}

良好的命名将改善代码的表现力,取好名字需要程序员良好的描述技巧和共有的文化背景。与其说它是一门技术,不如说它是一门艺术。必须坚持、并慎重地给程序中的每一个实体取好名字,让其名副其实,代码的可读性将得到大大地改善。

\end{content}

\section{Baby Names}
\begin{content}

\begin{regulation}
遵守统一的命名规范
\end{regulation}

\begin{table}[H]
\resizebox{0.95\textwidth}{!} {
\begin{tabular*}{1.2\textwidth}{@{}ll@{}}
\toprule
\ascii{Identifier} & \ascii{Examples} \\
\midrule
\ascii{Namespace}  & \ascii{std, dcm, mockcpp, testing} \\
\ascii{Class} & \ascii{Timer, FutureTask, LinkedHashMap, HttpServlet} \\ 
\ascii{Method} & \ascii{remove, ensureCapacity, getCrc} \\
\ascii{Constants} & \ascii{IDLE, ACTIVE} \\
\ascii{Local Variable} & \ascii{i, xref, houseNumber} \\
\ascii{Type Parameter} & \ascii{T, E, K, V, X, T1, T2} \\
\bottomrule
\end{tabular*}
}
\caption{命名规范}
\label{tbl:identifier-examples}
\end{table}

\begin{regulation}
类名应该是名词或名词短语;接口可以是形容词; 方法名应该是动词或动词短语; 返回值为\ascii{bool,加上is, has, can, should, need}将会增强语意。
\end{regulation}

以bool变量为例,如下例代码所示。

\begin{leftbar}
\begin{c++}
bool readPassword = true;
\end{c++}
\end{leftbar}

至少存在两种解释,

\begin{enum}
  \eitem{\ascii{We need to read the password}}
  \eitem{\ascii{The password has already been read}}
\end{enum}

\begin{leftbar}
\begin{c++} 
bool needPassword = true;
bool userIsAuthenticated = true; 
\end{c++}
\end{leftbar}

接口可以是形容词,最为常见的就是定义能力接口,即以\ascii{-able}结尾。
\begin{leftbar}
\begin{c++}
#include "base/Keywords.h"

INTERFACE(Runnable)
{
    virtual void run() = 0;
}
\end{c++}
\end{leftbar}

\begin{regulation}
丰富你的单词库,在面对具体问题时你具有更多的\ascii{Colorfull Words}.
\end{regulation}

如\reftbl{colorful-words}所示。

\begin{table}[H]
\resizebox{0.95\textwidth}{!} {
\begin{tabular*}{1.2\textwidth}{@{}ll@{}}
\toprule
\ascii{Word} & \ascii{Alternatives} \\
\midrule
\ascii{send}  & \ascii{deliver, dispatch, announce, distribute, route} \\
\ascii{find} & \ascii{dsearch, extract, locate, recover} \\ 
\ascii{start} & \ascii{launch, create, begin, open} \\
\ascii{make} & \ascii{create, set up, build, generate, compose, add, new} \\
\bottomrule
\end{tabular*}
}
\caption{Colorfull Words}
\label{tbl:colorful-words}
\end{table}

\begin{regulation}
程序中每个实体都应该有一个\ascii{Intention-Revealing}名称。
\end{regulation}

\begin{leftbar}
\begin{c++}
vector<vector<int> > getThem() {
    vector<vector<int> > list1;
    for (auto x : theList)
      if (x[0] == 4) 
        list1.add(x);
    return list1;
}
\end{c++}
\end{leftbar}

\begin{enum}
  \eitem{\ascii{vector<vector<int> >}的语法令人抓狂}
  \eitem{\ascii{getThem}让人看不清楚它的本意}
  \eitem{\ascii{theList}到底是什么东西?}
  \eitem{0下标意味着什么?4又意味着什么?}
  \eitem{\ascii{list1}就是为了编译通过吗?}
\end{enum}

换一个好名字之后,并对数据结构进行了简单的封装。

\begin{leftbar}
\begin{c++}
#include <vector>

struct Cell
{
    bool isFlagged() const;

private:
    vector<int> states;
};

typedef vector<Cell> Cells;

\end{c++}
\end{leftbar}

重构之后的效果,抽象和可读性得到了改善。

\begin{leftbar}
\begin{c++}
Cells getFlaggedCells() {
    Cells flaggedCells;
    for (auto cell : gameBoard)
      if (cell.isFlagged())
        flaggedCells.add(cell);
    return flaggedCells;
}
\end{c++}
\end{leftbar}

注:本例使用了\ascii{C++ 11}的\ascii{for-each}语法,简化迭代器的操作。

\begin{regulation}
使用注释不如花费更多时间给实体取一个好名字。
\end{regulation}

\begin{leftbar}
\begin{c++}
int time; // elapsed time in days
\end{c++}
\end{leftbar}

\begin{leftbar}
\begin{c++}
int elapsedTimeInDays;
\end{c++}
\end{leftbar}

注释应成为一种羞耻的活动,当需要添加注释的时候,往往是改善设计的最好时机。

但这不代表所有的代码都不写注释,即使我更加习惯用代码阐述意图,但有时也会增加必要的注释使代码更加容易被别人理解和维护。

当在操作位运算时,提供比特位的内存映像;定义正则表达式时,提供更直观的格式说明,将大大改善代码的可读性;另外,如果在代码中存在特殊的陷阱、实现手法等情况,此时注释变得尤为宝贵。

\begin{leftbar}
\begin{c++}
// Fast version of "hash = (65599 * hash) + c"
hash = (hash << 6) + (hash << 16) - hash + c;
\end{c++}
\end{leftbar}

\begin{regulation}
避免在名称中携带数据结构的信息
\end{regulation}

别用\ascii{accountList,accountArray}指定一组账号,当包含\ascii{Account}的容器不在是一个\ascii{List}或\ascii{Array}的时候,就会引发错误的判断。所以,用\ascii{accountGroup},\ascii{bunchOfAccounts},甚至直接使用\ascii{accounts},情况都会更好一些。

\begin{regulation}
\ascii{Noise Words are Redundant},消除噪声后将得到一个更加精准的名字。
\end{regulation}

\begin{table}[H]
\resizebox{0.95\textwidth}{!} {
\begin{tabular*}{1.2\textwidth}{@{}ll@{}}
\toprule
\ascii{Short Name} & \ascii{Redundant Names} \\
\midrule
\ascii{Name}  & \ascii{StrName, NameString} \\
\ascii{Customer} & \ascii{CustmerObject, CustmerInfo} \\ 
\ascii{accouts} & \ascii{accountList, accountArray} \\
\ascii{accout} & \ascii{accountData, accountInfo} \\  
\ascii{money} & \ascii{moneyAmount} \\
\ascii{message} & \ascii{theMessage} \\
\bottomrule
\end{tabular*}
}
\caption{Noise Words are Redundant}
\label{tbl:redundant-words}
\end{table}

\begin{regulation}
使用\ascii{Domain}的名称,将直白地表明你的设计
\end{regulation}

\begin{enum}
  \eitem{\ascii{Factory, Visitor, Repository}}
  \eitem{\ascii{valueOf, of, getInstance, newInstance, newType}}
  \eitem{\ascii{AppendAble, Closeable, Runnable, Readable, Invokable}}
\end{enum}

\end{content}

\section{Hungarian Notation}
\begin{content}

\begin{regulation}
摒弃匈牙利命名
\end{regulation}

匈牙利命名曾风靡一时,但是,现代编程语言具有更丰富的类型系统;人们更趋于使用更小的类,更短的函数,让每一个变量定义都在视野范围之内;\ascii{IDE}变得越来智能和强大,匈牙利命名反而变成了一种噪声和肉刺。

\begin{regulation}
摒弃给成员变量加前缀,或后缀
\end{regulation}

为类的成员变量增加\ascii{m\_}前缀是为了让成员变量与普通的变量有所区别,尤其在巨类时,\ascii{m\_}前缀可能有存在的必要。但与其写形如\ascii{m\_name}的成员变量,还不如花费更多的时间将类分解;当类足够小,职责单一,使用现代\ascii{IDE}的着色,前缀就像一根刺,让你的眼睛不舒服。

\begin{regulation}
摒弃接口和类的前缀
\end{regulation}

\begin{leftbar}
\begin{c++}
#include "base/Keywords.h"
#include "base/Status.h"

struct TransactionInfo;

INTERFACE(IAction)
{
    virtual Status exec(const TransactionInfo&) = 0;
}
\end{c++}
\end{leftbar}

\ascii{Action}前面的\ascii{I}就是一句废话。如果非得在接口与实现中选择,宁愿选择实现中增加\ascii{Impl}后缀。

\end{content}
