\begin{savequote}[45mm]
\ascii{I'm not a great programmer; I'm just a good programmer with great habits.}
\qauthor{\ascii{- Kent Beck}}
\end{savequote}

\chapter{简化逻辑}
\label{ch:simple-logic}

\section{简化控制逻辑}

\begin{content}

\begin{regulation}
\ascii{YODA Notation}已经是过去时
\end{regulation}

\begin{leftbar}
\begin{c++}
if (length <= MAX_STREAM_LEN)
if (MAX_STREAM_LEN >= length)
\end{c++}
\end{leftbar}

前者更具有表达力,再看一个例子,

\begin{leftbar}
\begin{c++}
while (bytesReceived < bytesExpected)
while (bytesExpected > bytesReceived)
\end{c++}
\end{leftbar}

同样地,前者更具有表达力。因为前者更加符合人类的思维,或这说更符合英语的表达习惯,\ascii{"if you are at least 18 years old."}明显比\ascii{"if 18 years is less than or equal to your age"}更加符合英语表达习惯。是的,英语更加习惯将稳定的一侧放在右边\ascii{(right-hand side)}。

但是,在\clang{}或者\cpp{}语言中,如果\ascii{==}被误用为\ascii{=},则可能发生副作用的危险,而且编译时并不能做到安全的检查,从而产生了严重的\ascii{bug}。

\begin{leftbar}
\begin{c++}
// 缺少等号,造成严重的运行时错误
if (ptr = NULL)
\end{c++}
\end{leftbar}

幸运的是,现代编译器对此类误用通常报告警告信息;此外如果保持\ascii{TDD}开发节奏,小步前进,此类低级错误很难逃出我们的法眼。

\begin{regulation}
在简单逻辑的情况下,鼓励使用\ascii{?:}三元表达式;但是,当表达式变得非常复杂时,\ascii{if-else}可能是更好的选择
\end{regulation}

反例:
\begin{leftbar}
\begin{c++}
if (hour >= 12) 
{
    time += "pm";
} 
else 
{
    time += "am";
}
\end{c++}
\end{leftbar}

正例:
\begin{leftbar}
\begin{c++}
time_str += (hour >= 12) ? "pm" : "am";
\end{c++}
\end{leftbar}

但很多程序员往往因为习惯了\texttt{if-else}而忽视了这种简洁的表达式。但是,当表达式变得非常复杂时,\texttt{if-else}可能是更好的选择。

反例:
\begin{leftbar}
\begin{c++}
return exponent >= 0 ? mantissa * (1 << exponent) : mantissa / (1 << -exponent);
\end{c++}
\end{leftbar}

正例:
\begin{leftbar}
\begin{c++}
if (exponent >= 0)
{
    return mantissa * (1 << exponent);
}
else
{
    return mantissa / (1 << -exponent);
}
\end{c++}
\end{leftbar}

\begin{regulation}
当逻辑表达式已经具有\ascii{true或false}语义时,则无需显示地加上\ascii{true或false}
\end{regulation}

反例:
\begin{leftbar}
\begin{c++}
bool exist = erabs.contains(ErabId(2)) ? true : false;
if(exist == true)
{
    ...
}
\end{c++}
\end{leftbar}

正例:
\begin{leftbar}
\begin{c++}
if(erabs.contains(ErabId(2))
{
    ...
}
\end{c++}
\end{leftbar}

\begin{regulation}
避免函数嵌套过深,擅用卫语句从函数中提前返回,或从嵌套的语句中提前返回;但绝对不滥用\ascii{break, continue}
\end{regulation}

嵌套过深的函数,将迫使读者跟踪运行时\ascii{stack},同样给读者增加过重的记忆包袱。

反例:
\begin{leftbar}
\begin{c++}
if (userResult == SUCCESS)
{
    if (permissionResult != SUCCESS)
    {
        reply.writeError(permissionResult);
        return;
    }
    reply.writeError("");
}
else
{
    reply.writeError(usrResult);
}
\end{c++}
\end{leftbar}

反例:
\begin{leftbar}
\begin{c++}
if (userResult != SUCCESS)
{
    reply.writeError(usrResult);
    return;
}

if (permissionResult != SUCCESS)
{
    reply.writeError(permissionResult);
    return;
}

reply.writeError("");

\end{c++}
\end{leftbar}

虽然在代码阅读过程中\ascii{break, continue,
return}常常打断了人的思维。但在函数短小,意图明确的情况下,\ascii{break, continue,
return}相反能给人一种“跳过此项”的特殊意图,代码嵌套逻辑得到了改善。

\begin{regulation}
擅用解释型变量,或提供意图明确的、内联的提取函数
\end{regulation}

解释型变量,或提取函数的重构手法有利于改善代码的可读性,往往后者更具有表达力。

反例:
\begin{leftbar}
\begin{c++}
if (line.split(":")[0].trim() == "root")
\end{c++}
\end{leftbar}

正例:
\begin{leftbar}
\begin{c++}
String userName = line.split(":")[0].trim();
if (userName == "root")
{
    ...
}
\end{c++}
\end{leftbar}

\ascii{split}函数将字符串按照指定的正则表达式进行拆分,并提供一个良好的意图变量,将大大改善逻辑的清晰度。

\begin{regulation}
缩小变量的作用域
\end{regulation}

\begin{regulation}
将局部变量的作用域最小化
\end{regulation}

这条规则可能对从\clang{}转变过来的\cpp{}程序员更有指导意义,但改变这样的陋习可能需要一些适应的时间。例如局部于\ascii{for}的循环变量,当它的使命完成之后,应立即销毁它,避免这个变量被再次使用的风险。

\begin{leftbar}
\begin{c++}
map<string, string> addressBook;
for (auto &entry : addressBook) 
{ 
    std::cout << entry->first << "," << entry->second << std::endl;
}
\end{c++}
\end{leftbar}

同样的规则也使用与其他常见的情况,例如\ascii{if}语句,简化空指针的判断,简洁有效。

\begin{leftbar}
\begin{c++}
if (PaymentInfo* info = database.readPaymentInfo()) 
{
    std::cout << "User paid: " << info->amount() << std::endl;
}
\end{c++}
\end{leftbar}

但也有人对于\ascii{if}语句这样的用法持反对态度,因为他们害怕这样做更容易出错。但在最坏的情况下,即使误写成了\ascii{==},也只是编译时错误。

\end{content}
