\begin{savequote}[45mm]
\ascii{Controlling complexity is the essence of computer programming.}
\qauthor{\ascii{- Brian Kernighan}}
\end{savequote}

\chapter{不可变性}
\label{ch:immutability}

\begin{content}
我们无时无刻都在面临着新的变化,优秀的程序员擅于控制变化,尽量让对象变得不可变\ascii{(immutable)},以便降低问题的复杂度。
\end{content}

\section{const}

\begin{content}

\begin{regulation}
使用\ascii{const}替代\ascii{\#define}宏常量定义
\end{regulation}

使用\ascii{const}替代宏常量理由在经典的《\ascii{Effective \cpp{}}》著作中进行了详尽的描述,但往往受\clang{}语言根深蒂固的习惯常常被人遗忘。

\begin{regulation}
当成员函数具有查询语义时应声明为\ascii{const}成员函数
\end{regulation}

此处具有查询语义的函数,泛指所有对对象状态未产生副作用的函数。\ascii{const}成员函数往往都以\ascii{get, is, has,
should}开头。

反例:
\begin{leftbar}
\begin{c++}[caption={math/Rectangle.h}]
#ifndef EWRIUN9037_NVHD6452_JKLSDFUIE7562_HGYE
#define EWRIUN9037_NVHD6452_JKLSDFUIE7562_HGYE

#include "base/BaseTypes.h"

struct Rectangle
{
    Rectangle(const U8 width, const U8 height);

    U16 getArea();
    U16 getPerimeter();

private:
    const U8 width;
    const U8 height;
};

#endif
\end{c++}
\end{leftbar}

此例中未对那些具有查询语义的函数声明为\ascii{const}。\ascii{const}的存在不仅仅是为了提高编译时的安全性检查,更重要的是\ascii{const}建立了用户和类之间的契约关系,并传达了程序员良好设计的心声。

当对象以\ascii{pass-by-const-reference}传递时,用户只能调用其\ascii{const}成员函数;此外,\ascii{const}建立的契约,使\ascii{Replace Temp with Query}成为可能。

正例:
\begin{leftbar}
\begin{c++}[caption={math/Rectangle.h}]
#ifndef EWRIUN9037_NVHD6452_JKLSDFUIE7562_HGYE
#define EWRIUN9037_NVHD6452_JKLSDFUIE7562_HGYE

#include "base/BaseTypes.h"

struct Rectangle
{
    Rectangle(const U8 width, const U8 height);

    U16 getArea() const;
    U16 getPerimeter() const;

private:
    const U8 width;
    const U8 height;
};

#endif
\end{c++}
\end{leftbar}

\begin{regulation}
\ascii{const}成员函数不应该修改系统的状态
\end{regulation}

编译器只能保证\ascii{const}成员函数修改成员变量时提示错误,而未对修改全局变量,类的静态变量的情况进行检查,应杜绝此类情况的发生。

\begin{regulation}
以\ascii{pass-by-reference-to-const}替代\ascii{pass-by-value},以改善性能,并避免切割问题
\end{regulation}

虽然我们不提倡进行过早的优化,但也绝不提倡不成熟的劣化。\ascii{pass-by-reference-to-const}就是此句名言最好的一个例子。

\begin{advise}
当传递内置类型,迭代器及函数对象时,则可以\ascii{pass-by-value}。如果使用\ascii{const}修饰,往往可以改善\ascii{API}的可读性
\end{advise}

其中,\ascii{Status, ActionId}分别是\ascii{U32,
U16}的\ascii{typedef},但在设计\ascii{onDone}原型时,\ascii{const}的修饰不仅仅为了改善其安全性,更重要的是与\ascii{TransactionInfo}的\ascii{reference-to-const}形成对称性\footnote{关于对称性,请参考\ascii{Kent
Beck}的著作\ascii{《实现模式》}},改善了\ascii{API}的美感。

反例:
\begin{leftbar}
\begin{c++}[caption={trans-dsl/listener/TransactionListener.h}]
#ifndef UTJKLFJS_467867_NBHD83562_HETRIO_75649
#define UTJKLFJS_467867_NBHD83562_HETRIO_75649

#include "base/Status.h"
#include "trans-dsl/concept/ActionId.h"

struct TransactionInfo;

struct TransactionListener
{
    virtual Status onDone(const TransactionInfo&, const ActionId&, const Status&) 
    {} 
};

#endif
\end{c++}
\end{leftbar}

正例:
\begin{leftbar}
\begin{c++}[caption={trans-dsl/listener/TransactionListener.h}]
#ifndef UTJKLFJS_467867_NBHD83562_HETRIO_75649
#define UTJKLFJS_467867_NBHD83562_HETRIO_75649

#include "base/Status.h"
#include "trans-dsl/concept/ActionId.h"

struct TransactionInfo;

struct TransactionListener
{
    virtual Status onDone(const TransactionInfo&, const ActionId, const Status) {}
};

#endif
\end{c++}
\end{leftbar}

\begin{regulation}
当返回的是一个新创建的对象时,必须返回其值类型; 更有甚者,返回\ascii{const}的值类型将改善编译时的安全性
\end{regulation}

反例:
\begin{leftbar}
\begin{c++}[caption={domain/GlobalIdentity.h}]
#ifndef UTYKLFLDQX74697_KJTUIONQA_3895903_NBFIT
#define UTYKLFLDQX74697_KJTUIONQA_3895903_NBFIT

#include "base/BaseTypes.h"

struct GlobalIdentity
{
    explicit GlobalIdentity(const U16 id);

    bool isValid() const;

    bool operator==(const GlobalIdentity& rhs) const;
    bool operator!=(const GlobalIdentity& rhs) const;

    static GlobalIdentity& valueOf(const U16 value);
	
private:
    const U16 id;	
};

#endif
\end{c++}
\end{leftbar}

静态工厂方法\ascii{valueOf}返回的是一个新创建的实例,所以返回值必须设计为值类型,提供\ascii{const}的修饰能够加强编译时的安全性检查。

正例:
\begin{leftbar}
\begin{c++}[caption={domain/GlobalIdentity.h}]
#ifndef UTYKLFLDQX74697_KJTUIONQA_3895903_NBFIT
#define UTYKLFLDQX74697_KJTUIONQA_3895903_NBFIT

#include "base/BaseTypes.h"

struct GlobalIdentity
{
    explicit GlobalIdentity(const U16 id);

    bool isValid() const;

    bool operator==(const GlobalIdentity& rhs) const;
    bool operator!=(const GlobalIdentity& rhs) const;

    static const GlobalIdentity valueOf(U16 value);

private:
    const U16 id;	
};

#endif
\end{c++}
\end{leftbar}

\begin{regulation}
不能返回局部对象的引用或指针。
\end{regulation}

返回值类型还是引用类型常常会困扰部分\cpp{}程序员。其实规则非常简单,返回值对象当且仅当需要创建新对象的时候,此时如果企图返回引用或指针,将导致运行时异常。

\end{content}

\section{不可变类}

\begin{content}

\begin{advise}
鼓励设计不可变类\ascii{(Immutable Class)}表达值对象\ascii{(Value Object)}的语义
\end{advise}

不可变类的每一个实例在创建的时候就包含了所有必要的信息,其对象在整个生命周期中都不会发生变化。不可变类具有如下几方面的优势:
\begin{enum}
  \eitem{相对于可变对象的复杂的状态空间,不可变类仅有一个状态,容易设计、实现和控制}
  \eitem{不可变类本质上是线程安全的,它们不需要同步}
  \eitem{不可变类可被自由地共享,重用对象是一种良好设计的习惯}
\end{enum}

设计不可变类,需遵循如下规则:
\begin{enum}
  \eitem{不能提供修改对象状态的方法}
  \eitem{保证类不能被扩展,切忌提供\ascii{virtual destructor}}
  \eitem{所有域都是\ascii{const}}
  \eitem{所有域都是\ascii{private}}
  \eitem{绝不允许重新赋值}
\end{enum}

不可变类唯一的缺点就是,对于每一个不同的值都需要一个单独的对象。当需要创建大量此类对象时,性能可能成为瓶颈。当性能成为关键瓶颈的时候,为不可变类设计可变配套类是一种典型的设计方式。

例如\ascii{Java}的\ascii{JDK}中,\ascii{String}被设计为不可变类,但在诸如字符串需要大量的替换、追加等场景下,性能可能成为关键瓶颈;\ascii{JDK}提供了\ascii{StringBuilder}的可变配套类来解决此类问题。

上例讲解的\ascii{Rectangle}类也是一种典型的不可变类的设计。

\begin{regulation}
尽量使类的可变性最小化
\end{regulation}

坚决抵制为每一个\ascii{get}函数都写一个\ascii{set}函数,除非存在一个很好的理由。构造函数的职责就是来完成对象的初始化,并建立起所有的约束关系。除非有很好的理由,不要在构造函数之外提供其他公开的初始化方法。

\end{content}

